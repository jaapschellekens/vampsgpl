% -*- LaTeX -*-

\part{Model description}

\chapter{Introduction}
This part of the manual will describe the principles on which \vamps\
was based.  If things are not explained here a reference to existing
literature is given.  If you think the manual is incorrect it might be
out of date as \vamps\ is still being developed. The best thing is to
check the source code if you have any doubts.

The main goals when the development of \vamps\ started were: {\em (1)}
make a parameterized model which can adequately describe the flow of
water in a forested environment (on a plot base) and {\em (2)} be
flexible enough to be able to be applied to the variety of forested
environments that exist.  However, because \vamps\ is very flexible it
is also possible to apply \vamps\ to plots covered with agricultural
crops or bare soils.  Also {\em (3)} it is the authors opinion that
(modelling) software should be distributed with complete source code
and documentation to allow the user to adapt the program to any new
situation and examine the program's internal workings.  

The flexible nature of \vamps\ can be a bit confusing at first because
of the large amount of choices to be made. However, this manual and
the example files should be able to guide you to a successful
application of the \vamps{} model.

The flow of water through a forested ecosystem is described in
Figure~\ref{fig:fcycle}. In general three pathways can be distinguished
by which precipitation reaches the forest floor. A small fraction of
the water reaches the forest floor without touching leaves or stems --
known as direct throughfall. Another -- also small -- fraction of the
water flows down the tree trunks as stemflow. The rest of the water
hits the forest canopy and will leave this canopy as crown drip or
evaporation from the wet canopy depending on canopy storage and shape,
kinetic energy of the droplets and the athmosphere's evaporative demand.
Water that infiltrates in the soil profile can leave this laterally as
saturated or unsaturated flow, percolate to deep groundwater or it can
be extracted by the roots of plants or trees. If the troughfall intensity
exceeds the infiltration capacity of the topsoil Hortonian overland flow
can occur. The top layer of the profile can become saturated resulting
in Saturation overland flow. \vamps\ simulates most of these fluxes
as can be seen in Figure~\ref{fig:vflow} which shows a simplified flow
diagram of the \vamps\ model. The model is mainly driven by input from the
atmosphere\footnote{As the bottom boundary conditions can also change in
time one could probably regard this as a driving force as well}. In the
following chapters the workings of \vamps\ and how the model simulates
each of these fluxes will be explained.

It should be noted that most fluxes in the model can either be
calculated or given by the user. The atmosphere, canopy and soil part
of the model can be combined or used separately. However, the strength
of models like \vamps\ is their ability to combine all these to an
integrated model of the forest hydrological cycle. The canopy part in
particular is closely interwoven with the atmosphere module and is
hard to use seperate from the atmosphere part. 


%-------------------------------------------------------------------
% This chapter should summerize the functions of the main modules
% Finish this first.

\chapter{The modules}

\section{Introduction}
\vamps\ is set up is such a way that parts of the model can be used
or disregarded depending on the user's needs and the available data.
This implies that you don't need {\em all} the following information.
Although the setup of the following chapters might suggest a strict
separation between the modules parts overlap and parts cannot be used
without some parts of other modules.

\section{The atmosphere}
In the atmosphere module, potential transpiration, potential soil
evaporation and evaporation from a wet surface are calculated from
meteorological and vegetation parameters. Several methods can be
chosen, depending on the data available and the user's
preferences. The terms potential evaporation, actual evaporation and
evapotranspiration can lead to some confusion, particularly with the
use of models like \vamps{}. Also the terms can have slightly
different meanings when different methods are used. Although this is
hardly desirable \vamps{} is no exception yet. In the final version of
the model I hope to simplify all this, in the mean time most names and
units remain as quoted from the original source.

%If you want to model a canopy in detail (see section~\ref{sec:canopy}) some
%of the stuff presented in this module is not needed.


\subsection{Potential evaporation}
Potential evaporation can be used in \vamps{} to determine actual
evaporation (or evapotranspiration, depending on the method chosen) in
an other part of the model. Three methods can be used:

\begin{enumerate}
\item Determine Penman \Index{$E_0$} (open water evaporation)
\cite{penman1956N,makkink1957} using reflected
radiation, net-radiation, relative humidity (this is used to calculate
the vapour pressure deficit), wind-speed, temperature and incoming
radiation.
\item Determine Penman $E_0$ using sun-ratio, relative humidity, wind-speed,
temperature and incoming radiation.
\item Using the Makkink formula \cite{makkink1961,commissie1988N} which needs
incoming radiation, relative humidity, wind-speed and temperature.
\end{enumerate}

The input file settings for potential evaporation are described in
section~\ref{section:pevaporation} on
page~\pageref{section:pevaporation}.

Penman $E_0$ is determined via:
\begin{equation}
E_0 = \frac{\Delta R_{no} + \gamma E_a}{\Delta + \gamma}
\end{equation}
\begin{where}
$R_{no}$& net radiation over open water [cm/day]\\
\Index{$E_a$}	& aerodynamic evaporation or drying power of air [cm/day]\\
$E_0$ 	& open water evaporation [cm/day]\\
\Index{$\Delta$} & slope of the saturation vapour pressure curve\\
\Index{$\gamma$} & psychrometer 'constant' [$mbar/{}^o K$]\footnote{An
air-pressure of 998 mbar is assumed in calculating $\gamma$}\\
\end{where}

The determination of $E_a$
is described in \cite{calder1990284}:

\begin{equation}
E_a = 2.6  (e_s - e_a)  (1 + 0.537 u)
\end{equation}
\begin{where}
\Index{$e_a$}	&	actual vapour pressure [mbar]\\
\Index{$e_s$}	&	vapour pressure at saturation [mbar]\\
\Index{$u$}	&	mean daily wind-speed at 2m\\
\end{where}
\vamps\ calculates $e_s$ and $e_a$ from relative humidity and
dry bulb temperature using an equation 
described by~\citeasnoun{bringfelt86}.

Net radiation over open water is given by:
\begin{equation}
R_{no} = R_s (1- \alpha) - R_{nl}
\end{equation}
\begin{where}
\Index{$R_s$}		& incoming solar radiation \\
\Index{$R_{nl}$}	& net long-wave radiation \\
\Index{$\alpha$}  	&albedo of open water (0.05) \\
\end{where}

In method 1 $R_{nl}$ is calculated using incoming and
reflected short-wave radiation to calculation net long-wave
radiation via:
\begin{equation}
 R_{nl} = R_s - R_{net} - Rs_{out}
\end{equation}
\begin{where}
$R_s$ 			& incoming solar radiation \\
\Index{$Rs_{out}$} 	& reflected short-wave radiation \\
\Index{$R_{net}$}	& net radiation\\
\end{where}
In method 2 sun-ratio $n/N$ is used to calculate $R_{nl}$ via:
\begin{equation}
R_{nl} = \frac{86400 \sigma T^4 (0.56 - 0.248 \sqrt e) (0.1 + 0.9 n/N)}{\lambda}
\end{equation}
\begin{where}
\Index{$T$} 	& mean daily air temperature [${}^o K$]\\
\Index{$e$}	& mean daily water vapour pressure [kPa]\\
\Index{$n/N$}	& \parbox[t]{7cm}
{ratio of duration of bright sunshine hours, n, 
	to the maximum possible duration of sunshine hours, N} \\
\Index{$\lambda$} & latent heat of vaporization \\
\end{where}


Makkink~\cite{commissie1988N} reference evaporation is calculated by:
\begin{equation}
\lambda E = C \frac{\Delta}{\Delta + \gamma} R_s
\end{equation}
\begin{where}
\Index{$C$} 	& Constant (usually 0.65)\\
$\lambda, E, \Delta, \gamma, R_s$ & are defined as before\\
\end{where}


\subsection{Actual evapotranspiration}
If you don't want to model an entire canopy (as is described in
section \ref{sec:canopy}) you can estimate actual evapotranspiration
with one the following three methods. The use of the canopy section
is the preferred way.

\begin{enumerate}
\item Set actual evapotranspiration to be equal to potential evaporation.
\item Multiply potential evaporation by a crop factor
\item Calculate actual evaporation using the Penman-Monteith formula
\end{enumerate}

%\vamps\ of course needs to now how much of this amount is attributed
%to soil evaporation, transpiration and evaporation from a wet canopy.
%The variables {\tt int\_frac}, {\tt sevap\_frac} and {\tt trans\_frac}
%determine this. Note that an approach like this only makes sense of you
%use daily or larger time-steps.



\section{Plants and trees}\label{sec:canopy}

Although \vamps\ has been set up to be able to use a multilayer
canopy, this option is untested and has been disabled in the alpha
version. At present you are stuck to one canopy layer.

At the base of this module lies the Penman-Montheith equation\cite{monteith1965}.


\begin{equation}
\lambda ET = \frac{\Delta (R_n - G) + \rho C_p \delta e/r_a}
{\Delta + \gamma ( 1 + r_s/r_a)}
\end{equation}

\begin{where}
$\rho$	& density of air \\
$c_p$	& specific heat of air \\
$\delta e$	& vapour pressure deficit \\
$\gamma$	& psychrometic constant \\
$\Delta$	& the slope of the saturation vapour pressure curve \\
$r_s$	& canopy resistance \\
$r_a$	& aerodynamic resistance\\
\end{where}

The program seperates wet and dry periods.  Evaporation from the wet
canopy is calculated using the Penman-Monteith equation with $r_s$ set
to zero.  During these times transpiration (and thus rootextraction)
is assumed zero.

The amount of net radiation available is determined via a simple
fraction or via an equation by \citeasnoun{roberts1993197} linking the
leaf area profile to absorbed radiation.

This module combines most of what has been explained before (in the
sections concerning evaporation, transpiration and interception) and
\vamps\ work best if you use this.


\subsection{Interception}\label{sec:interception}

\paragraph{Introduction}\footnote{The settings for the input file that control interception are described in section~\ref{section:interception} on
page~\pageref{section:interception}.}
\Index{Interception} of precipitation can comprise a large part of the water
balance in forested catchments.  Interception is defined as
\cite{leonard1967184}:
\begin{quote}
\ldots the amount of rainfall retained by the vegetation and evaporated
without dripping off or running down the stem. At some point after 
precipitation ceases, all of the intercepted water will be evaporated\ldots
\end{quote}
Note that losses in the litter layer are not included\footnote{\vamps
, the final version, can estimate litter interception as a seperate
term}.  The balance between precipitation, evaporation and canopy
storage\footnote{Although the word canopy implies a forest most of the
methods described here can also be applied to agricultural crops or
grassland} \cite{rutter1971174} determines the amount of rainfall
interception.  As the canopy parameters can change with vegetation
type, so does the interception.

Interception of a forest canopy is generally determined by either the
water budget method or by the aerodynamic method
\cite{tsukamoto1989185}.  An alternative method consists of hanging a
tree in its place in the forest and weighing it continuously. This has
the advantage that the evaporation during the storm can adequately
be determined \cite{tsukamoto1989185}.

In tropical rain forest areas a great diversity of species exist. Highly
significant differences in interception storage have been found 
\cite{leonard1967184,herwitz1985181,scatena1990158}. This implies that
parameters probably need to be determined for each site seperately.

How interception losses can be calculated is mostly determined by the
available data. The model by \citeasnoun{rutter1971174} needs hourly
input, while some of it's parameters should be determined using 5 minute
values. In reality daily values are most commonly available, so
developing equations for daily values (despite the obvious limitation)
is justifiable and certainly practical \cite{jackson1975183}. 
\citeasnoun{gash1979165} developed an analytical
variant of the Rutter model which can be used with daily input values
assuming one storm a day.

Several methods for determining interception losses can be used within
the \vamps\ model. The choice of method is mainly governed by the
available input data. The following paragraphs present the methods
that \vamps\ can use to determine interception losses.

\paragraph{\Index{Gash}'s method}\label{par:gash}
If input data is available on a daily basis, the method described
by~\citeasnoun{gash1979165} is the preferred method.  Gash's model is
based on a numerical model developed by \citeasnoun{rutter1971174}. In
contrast to Rutter's model which is described later, it can be applied
to daily measurements assuming one storm a day. \citeasnoun{gash1980166}
have tested both models on the same data set and found comparable
results.

In the model  \Index{$P$'} -- the  amount  of water  needed to completely
saturate the canopy \cite{gash1979165} -- is defined as:

\begin{equation}
P\acute{}=\frac{-\overline{R}S}{\overline{E}}
ln\left[1-\frac{\overline{E}}{\overline{R}}
(1-p-p_{Tr})^{-1}\right] 
\end{equation}

Interception losses  for storms larger and  smaller than  $P\acute{}$ are dealt
with separately. For small storms the equation becomes:

\begin{equation}
E_i = (1-p-p_{Tr}) P 
\end{equation}
while or large storms it becomes:
\begin{equation}
E_i = (1-p-p_{Tr}) P\acute{} + (\overline{E}/\overline{R}) (P - P\acute{})
\end{equation}

\begin{where}
\Index{$\overline{R}$}  & average precipitation [mm/d]\\
\Index{$\overline{E}$}  & average evaporation during storms [mm/d] \\
\Index{$S$} 	        & storage capacity of the canopy [mm] \\
\Index{$p$}             & free throughfall coefficient \\
\Index{$p_{Tr}$}        & stemflow coefficient \\
\Index{$P$}		& precipitation [mm/d]\\
\Index{$E_i$}		& interception loss [mm]\\
\end{where}

The interception losses from the stem are calculated for days that $P
\geq S_{Tr}/p_{Tr}$. As this is generally small (about 2\% of gross $P$)
it is often neglected\footnote{This is also true for \vamps}. \vamps\
can calculate The ratio of $\overline{E}/\overline{R}$ from the
Penman-Monteith equation or use a fixed value.\footnote{The latter is
generally preferable as the penman-Monteith equation (in this case
with $R_s$ set to zero) in not very good at predicting in storm
evaporation.  This seems to be due to the influence of advective
energy during a storm. Usually $\overline{E}/\overline{R}$ is
determined by a regression of interception losses versus gross
precipitation including only single storm events.} \vamps\ also allows
the parameters $S$, $p$ and $p_{Tr}$ to be linked to the Leaf Area
Index (LAI). If lai is defined in the \inisec{interception} section of the
input file, or in the \inisec{ts} section of the input file a method
described in~\citeasnoun{dijk1996N} is used to link these parameters to LAI.

A version of the Gash model adapted to small time-steps can also be
used in \vamps{}. In this version the canopy storage ($S$) is
diminished with the value of the actual canopy storage at that
moment. This new value is now used to calculate $P$' and interception
losses.  To use this method Evaporation during that time-step is
needed as well as this is subtracted from the amount of water in the
canopy at the begin of the time-step. Results of applying this adapted
version to a test case using the normal version with daily time-steps
and the adapted version with half-hourly data, show that interception
losses as modeled with both methods did not differ more than 2\%.

\paragraph{\Index{Calder regression model}}

\citeasnoun{calder1986171} presented a simple regression model for interception
loss. Despite it's simplicity it could adequately describe the interception
of forests in the UK. It is defined as:
\begin{equation}
E_i = \gamma \left[ 1 - exp (-\delta P) \right]
\end{equation}
\begin{where}
$E_i$			& Interception loss [mm/d] \\
\Index{$P$}		& precipitation [mm/d] \\
\Index{$\gamma$}	& maximum interception loss per day [mm] \\
\Index{$\delta$}	& fitting parameter \\
\end{where}



\paragraph{\Index{Interception as a fraction of LAI}}

In this method interception is calculated as:
\begin{equation}
E_i = min (P,laifrac*LAI) 
\end{equation}
\begin{where}
$E_i$			& Interception loss [mm/d] \\
$P$			& precipitation [mm/d] \\
\Index{$laifrac$}	& fraction (usually around canopy storage) \\
\Index{$LAI$}		& leaf area index \\
\end{where}
This method can describe total interception rather well but fails on a
storm to storm basis (it is used in the TOPOG model
\cite{vertessy1993140}). The fact that it is based on LAI makes it well
suited for applications in which the canopy changes in time, although
an alternative Gash model in which the parameters $S$, $p_{tr}$ and
$p$ are related to LAI might yield better results (see also
paragraph~\ref{par:gash} on page~\pageref{par:gash}).

\paragraph{\Index{Rutter's model}}
\citeasnoun{rutter1971174} (see also \citeasnoun{rutter1975172}) developed
a model of interception based on the storage of water in the canopy.
The model needs hourly meteorological data plus canopy parameters as
input, although five minute data are preferred.

\begin{equation}
dC/dt = P(1 - p - p_{Tr}) - (C/S) E_c - D_r
\end{equation}
\begin{where}
$t$			& time-step [hour]\\
\Index{$C$}		& amount of water in the canopy\\
$P$			& gross precipitation [mm/h] \\
$p$			& free throughfall coefficient [-] \\
$p_{Tr}$		& proportion of water discharged via stem flow\\
$S$			& canopy storage capacity [mm]\\
\Index{$E_c$}		& evaporation rate from wet canopy [mm/h]\\
\Index{$D_r$}		& drainage-speed from storage in canopy [mm/h] \\
\end{where}

The model is rather insensitive to both $S$ and $p$ \cite{gash1978167}.


\section{The soil} 
\subsection{Introduction}
The soil module can be regarded as the core of \vamps. The model was
developed using the {\sc swap94} model as a framework. About 80\% of
\vamps 's soil module consists of hand-translated\footnote{And changed
to my liking} Fortran code from the original {\sc swap94} program. The
head based solution of Richard's equation using remains more or less
the same in the present version, although the solution has been
extended with a band-diagonal procedure and a (optional) step to regain
full machine precision (described in
\citeasnoun{press1992B}) that can be used if a zero pivot
occurs in the tri-diagonal solution.


\subsection{Solving Richard's equation}
The soil module solves Richards equation using a $\psi$ based form of this
equation. A differential moisture capacity ($C(h)$) has been introduced to
have a single independent variable. This differential soil moisture capacity
is equal to the slope of the soil-moisture retension curve, $d\theta/d\psi$.

\begin{equation}
\frac{\partial \psi}{\partial t} = \frac{1}{C(\psi)}
\frac{\partial}{\partial z} \left[ K(\psi) \left(
\frac{\partial \psi}{\partial z} - 1 \right) \right]
\end{equation}

A sink term (S) has been added to accommodate water extraction by roots
and lateral flow.

\begin{equation}
\frac{\partial \psi}{\partial t} = \frac{1}{C(\psi)}
\frac{\partial}{\partial z} \left[ K(\psi) \left(
\frac{\partial \psi}{\partial z} - 1 \right) \right]
- \frac{S(\psi)}{C(\psi)}
\end{equation}

\begin{where}
$\psi$	& Suction head [cm] \\
t	& time [days] \\
$C$	& differential moisture capacity ($d\theta/d\psi$)\\
$S$	& sink term\\ 
$z$	& heigth [cm]\\
\end{where}

At present the reader is referred to \citeasnoun{feddes1978273} and
\citeasnoun{belmans1983272} for a detailed description.

\vamps\ first estimates the largest time-step possible with the code
presented in example~\ref{ex:timestep}. It uses the previous dt and
a value for thetol (the maximum change in theta allowed per time-step)
to do so. This is done for all layers in the soil profile. This way the
layer in which the largest changes occur determines the maximum dt.
\begin{Example}
\begin{verbatim}
/*-
 *	void timestep(int step)
 *	Calculation of timestep (dt) depending on theta  changes
 *	and new time (t)
 */
double
timestep (int step)
{
  int i;

  /* Store previous values */
  tm1 = t;
  dtm1 = dt;

  /* start with max dt */
  dt = dtmax;

  mdtheta = fabs (thetm1[0] - theta[0]);
  for (i = 0; i < layers; i++)
    {
      dtheta = fabs (thetm1[i] - theta[i]);
      mdtheta = mdtheta < dtheta ? dtheta : mdtheta;
    }

  if (mdtheta < 1.0E-6)
    dt = dtmax;
  else
      dt = 10.0 * thetol * dtm1 / mdtheta;

  /* restrict to dtmin and dtmax */
  dt = dt > dtmin ? dt : dtmin;
  dt = dt < dtmax ? dt : dtmax;

return dt;
}

\end{verbatim}
\caption{The C code used by \vamps\ to estimate the maximum dt}
\label{ex:timestep}
\end{Example}
If this produces results that are regarded as not accurate enough (set
with the \inivar{thetol} variable) \vamps\ will perform iterations to
get the accuracy asked. If this fails dt is halved and the procedure
is started again until the accuracy is reached or dt-minimum is
reached. The maximum number of iterations is set with \inivar{maxitr}
variable while the minimum allowed dt is set using the \inivar{dtmin}
variable.


\subsection{Relation between Theta, $\psi$  and  $K_{unsat}$} 
\vamps\ was first developed using the equations by 
\citeasnoun{genuchten1980179} (see also \citeasnoun{mualem1976261}).
This is still the default method. For extra speed and flexibility
\vamps\ can also read tables which describe these relations. This
means that the user can use any method available\footnote{the use of
lookup tables can in certain circumstances also cause some speed
increase}. At present \vamps\ can only generate tables for the van
Genuchten method. However, it can read tables generated by the {\tt
topog\_soil} program \cite{beverly1994274} or user made tables, thus
extending the amount of models for the theta vs $\psi$ that can be
used to anything the user can imagine.  The following is derived from
the topog\_soil\footnote{A version of topog\_soil for the
\vamps\ model will be available in the future} manual:

\begin{quote}
Six different soil tables can be generated using topog\_soil, the
options being:
\begin{enumerate}
\item Broadbridge and White (1988)
\item Campbell (Ross and Bristow, 1990)
\item Campbell (SWIM model version, 1974;1985)
\item Campbell (Ross, Williams and Bristow, 1991)
\item van Genuchten (1980)
\item Simplified Bucket Model soil table (1979)
\end{enumerate}
\end{quote}

At present only items 1 and 5 have been tested using \vamps .  A
description of the file format for these tables can be found in
Appendix ??.

\paragraph{Van Genuchten}

\vamps\ default and most tested method of describing the
theta versus suction- head and $k_{unsat}$ is base in the van
Genuchten equations \cite{mualem1976261,genuchten1980179}.  The
dimensionless water content is given as:



\begin{equation}
\Theta = \frac{\theta - {\theta}_r}{{\theta}_s - {\theta}_r}
\end{equation}
\begin{where}
$\theta_s$	& saturated water content \\
$\theta_r$	& residual water content\\
$\theta$	& actual soil water content\\
$\Theta$	& dimensionless soil water content\\
\end{where}

Then a class of $\Theta (\psi)$ functions can be given by:
\begin{equation}
\Theta = \left[ \frac{1}{1 + (\alpha \psi)^n}\right]^m
\end{equation}

\begin{where}
$\psi$ 	& pressure head \\
$\alpha$, $n$ and $m$	& soil characteristic parameters \\
\end{where}

if $m = 1 - n/m$ the following closed-for expression of $K(\Theta )$
can be obtained:

\begin{equation}
\Theta = \left\{ \begin{array}{ll}
		\frac{1}{[1+(\alpha \psi)^n]^{1-1/n}} 	& \psi < 0 \\
		1					& \psi \geq 0
		\end{array}
	\right.
\end{equation}

\begin{equation}
K = K_s \Theta^{1/2}
\left[ 1 - \left( 1 - \Theta^{n/(n-1)}\right)^{1-1/n}\right]^2
\end{equation}

The $ K_s \Theta^{1/2}$ part of the equation can be changed to $ K_s
\Theta^{L} $ is which $L$ is a fitting parameter describing the
$\Theta$ to $K_{unsat}$ relation if you have $K_{unsat}$ measurements.

Values for the parameters $\alpha$, $n$ and $L$ for most soil types in
the Netherlands can be found in \citeasnoun{wosten1987287}. Such
values for tropical soils seem to be less common. In that case a
nonlinear regression method can be used to determine the parameters
$\alpha$ and $n$ from a soil moisture retention curve
\cite{marquardt1963N}.

\paragraph{Clapp/Hornberger}
A power curve can conveniently describe the moisture characteristics
of a soil as has been demonstrated
by~\citeasnoun{campbell1974}. \citeasnoun{clapp1978263} described
these functions and presented some representative values for the soil
hydraulic parameters needed. They are shown in Table~\ref{tab:clapp}.

\begin{equation}
\psi = \psi{_s} \left(\frac{\theta}{\theta{_s}}\right)^{-b}
\end{equation}

Soil hydraulic conductivity functions are derived by integrating a
function of $\psi$ with respect to $\theta$. This results in the
following equation:
\begin{equation}
\frac{K}{K_s} = \left(\frac{\theta}{\theta{_s}}\right)^{2b + 2}
\end{equation}

\begin{where}
$\psi$	&	suction head \\
$\psi{_s}$	& suction head at saturation (air entry value) \\
$\theta$	& volumetric water content \\
$\theta{_s}$	& water content at saturation (porosity)\\
$b$	& soil characteristic parameter\\
$K$	& hydraulic cnductivity \\
$K_s$	& saturated hydraulic conductivity \\
\end{where}

\begin{table}
\centerline{
\begin{tabular}{lllll} \hline \hline
Soil Texture & $\overline{b}$ & $\overline{\psi{_s}}$ &  $\overline{\theta{_s}}$ & $\overline{K_s}$ \\ \hline
Sand & 4.05 & 12.2 & 0.395 & 1520.64 \\
Loamy sand & 4.38 & 9.0 & 0.410 & 1350.72 \\
Sandy loam & 4.90 & 21.8 & 0.435 & 299.52 \\
Silt loam & 5.3 & 78.6 & 0.485 & 62.21 \\
Loam & 5.39 & 47.8 & 0.451 & 60.05 \\
Sandy clay loam & 7.12 & 29.9 & 0.420 & 54.43 \\
Silty clay loam & 7.75 & 35.6 & 0.477 & 14.69 \\
Clay loam & 8.52 & 63.0 & 0.476 & 21.17 \\
Sandy clay & 10.4 & 15.3 & 0.426 & 18.72 \\
Silty clay & 10.4 & 49.0 & 0.492 & 8.93 \\
Clay & 11.4 & 40.5 & 0.482 & 11.09 \\ \hline \hline
\end{tabular}}
\label{tab:clapp}
\caption{Representative values for hydraulic parameters. 
After Clapp (1978).}
\end{table}
An input file containing all the soil-types mentioned in table~\ref{tab:clapp}
can be found in the src/sllib directory. It is called \fname{soillib.inp}.

\subsection{Bottom boundary conditions}
\vamps\ can  handle several boundary conditions at the bottom of  the
profile.  Below is a lists of those available in the model. Not all
conditions have been fully tested in the present version of \vamps{}.

\begin{itemize} 
\item A no-flow bottom boundary (5)
\item Free drainage at the bottom of the profile (6)
\item A fixed head at the bottom of the profile (4)
\item A fixed flux at the bottom of the profile (1)
\item Daily groundwater depth (0)
\item Seepage or infiltration from/to deep groundwater (2)
\item Flux calculated as a function of head (3)
\end{itemize}

The choice of bottom boundary condition is of great importance to
the results of the modelling. Unfortunately
most of the time the choice of bottom boundary is usually determined 
by the available data or lack thereof.

See {\bf vamps}(5) for details\footnote{The numbers between brackets
denote the value of the \inivar{bottom} variable in the \inisec{soil}
section needed}.
