% Generated by sldoc.slx (1.14, (C) 1996, 1997) Tue Dec 16 11:14:32 1997


\paragraph{\index{cat}cat}
\begin{verbatim}
Void cat(inf, outf)
\end{verbatim}
Copies the file {\tt inf} to {\tt outf} appending it if
{\tt outf} already exists. {\tt cat} only works with ascii
files.

from file: {\tt cat.sl}


\paragraph{\index{e0a}e0a}
\begin{verbatim}
Float e0a(Td,Rs,sunratio,u,Ea,Es,Slope,Gamma,L)
\end{verbatim}
Returns: e0 in mm$/$day (i think)
The determination of \$E\_a\$
is described in \verb1\1cite\{calder1990284\}:
\verb1\1begin\{equation\}
E\_a $=$ 2.6  (e\_s - e\_a)  (1 + 0.537 u)
\verb1\1end\{equation\}
\verb1\1begin\{where\}
\verb1\1Index\{\$e\_a\$\}	\&	actual vapour pressure [mbar]\verb1\1\verb1\1
\verb1\1Index\{\$e\_s\$\}	\&	vapour pressure at saturation [mbar]\verb1\1\verb1\1
\verb1\1Index\{\$u\$\}	\&	mean daily wind-speed at 2m\verb1\1\verb1\1
\verb1\1end\{where\}

from file: {\tt met.sl}


\paragraph{\index{e0b}e0b}
\begin{verbatim}
Float e0b(Td,Rs,Rsout,Rnet,u,Ea,Es,Slope,Gamma,L)
\end{verbatim}
Calculates penman open water evaporation using net radiation.
Use the {\tt delgam} function to get Gamma, Slope, and L


Returns: e0 in mm$/$day (i think)

from file: {\tt met.sl}


\paragraph{\index{eaes}eaes}
\begin{verbatim}
Float,Float eaes(td,rh)
\end{verbatim}
Determines saturation vapour pressure (ea) and actual vapour presssure 
(ea) from relative humidity and dry bulb temperature. es is calculated
according to Bringfeld 1986. Relative humidity should be in \% and
dry bulb temp in degrees Celcius. 
Example:
\begin{verbatim}
variable ea,es;

(ea, es) = eaes(20.0,76);
printf("ea = %f es = %f\n",ea,es,2);
\end{verbatim}
Returns: ea, es

from file: {\tt met.sl}


\paragraph{\index{gamma}gamma}
\begin{verbatim}
Float gamma(td)
\end{verbatim}
Calculates the psychometric constant ([\$mbar$/$\{\}\verb1^1o K\$])
It assumes an air pressure of 998mb. Cp is estimated
at 1005.0 J$/$kgK


Returns: psychrometric constant

from file: {\tt met.sl}


\paragraph{\index{makkink}makkink}
\begin{verbatim}
Float makkink(rad,Slope,Gamma)
\end{verbatim}
calculates reference evaporation according to Makkink
The C1 constant is taken as 0.65 and C2 0.0


Returns: makkink reference evaporation in mm$/$day

from file: {\tt met.sl}


\paragraph{\index{ra}ra}
\begin{verbatim}
Float ra (z, z0, d, u)
\end{verbatim}
Calculates ra (aerodynamic resistance) according to:
\begin{verbatim}
    ra = [ln((z-d)/z0)]^2 /(k^2u)
    Where:
        d  = displacement height
        z  = height of the wind speed measurements
        u  = windspeed 
        z0 = roughness length
        k  = von Karman constant
\end{verbatim}
Returns: ra

from file: {\tt met.sl}


\paragraph{\index{vslope}vslope}
\begin{verbatim}
Float vslope(Td,es);
\end{verbatim}
Calculates the slope of the saturation vapour pressure 
curve from es and dry bulb temp for use in Penman Eo and
Penman-Montheith.  
Required input: Td (dry bulb temp [oC]), es


Returns: slope of the vapour pressure curve

from file: {\tt met.sl}


\paragraph{\index{endsubplot}endsubplot}
\begin{verbatim}
Void endsubplot()
\end{verbatim}
Resets the plotting system to single plot mode. Also needed to actually
draw the plot on some terminals. See also {\tt subplot}.

from file: {\tt multplt.sl}


\paragraph{\index{pprint}pprint}
\begin{verbatim}
Void pprint()
\end{verbatim}
Print the current plot on a postscript printer

from file: {\tt multplt.sl}


\paragraph{\index{subplot}subplot}
\begin{verbatim}
Void subplot(x, y, index)
\end{verbatim}
Set the plotting system for multiple plots on one page. {\tt x} and {\tt y} are the
number of plots in the x and y direction respectively. {\tt index} denotes the
number of the subplot in which the next plot is drawn. Numbering starts
at one.  The first plot is drawn in the upper left corner, the last in the
lower right. Currently subplot only works with gnuplot. See also
{\tt endsubplot}.

from file: {\tt multplt.sl}


\paragraph{\index{f\_plot}f\_plot}
\begin{verbatim}
Void f_plot(String var,String  thef,Int xcol, Int ycol)
\end{verbatim}
Plots var {\tt var} from vamps output file {\tt thef}.
It does so by calling {\tt vsel} to construct a matrix. This
matrix is than passed to {\tt m\_plot} to make the actual plot.
Column 0 is time.
See also {\tt plot} and {\tt plot.sl}.

from file: {\tt plot.sl}


\paragraph{\index{m\_plot}m\_plot}
\begin{verbatim}
Void m_plot(Matrix m, Int xcol, Int ycol, String title, String xlab,
        String ylab)
\end{verbatim}
Plots an XY diagram of matrix {\tt m} using the current graphics method.
Column numbering for {\tt xcol} and {\tt ycol} starts at zero. Depending on the
graphics method used the top, x and y-axis labels strings cannot be empty.
example:
\begin{verbatim}
outputdev = 1; % screen graphics
m_plot(m,0,1,"Matrix m","Time","Q");
outputdev = 0; % Plot is OK, make hardcopy
m_plot(m,0,1,"Matrix m","Time","Q");
\end{verbatim}
See also {\tt plot} and {\tt plot.sl}.

from file: {\tt plot.sl}


\paragraph{\index{plot}plot}
\begin{verbatim}
plot.sl - high level plotting functions
\end{verbatim}
Plot functions to allow quick visualization of Vamps output files
and matrix variables. The syntax of the high level functions ({\tt f\_plot}
and {\tt m\_plot}) is the same, no matter which underlying graphics method
is used. Low level functions depend on the graphics method used. The
following graphics methods can be used:
\begin{verbatim}
    plot        - gnu plotutils lib 1.1
    agl        - AGL graphics lib 3.?
    ext_plot    - using external gnu plotutils programs (UNIX only)    
    gnu_plot    - using gnuplot interface (UNIX only)    
              see also @#@ and @plt@.    
\end{verbatim}
Which of these methods can be used in this version of vamps is
printed to the screen when using the {\tt plot} command to initialize the
plot system. The {\tt show\_plotsys()} function can be used for this purpose
as well.


The {\tt outputdev} variable can be used to specify screen preview (1) or
hardcopy output (0).


If more than one graphics method is available you can switch between 
methods be setting the {\tt m\_plot} function to the method specific function.
Use the {\tt set\_plotsys} function to do so.

from file: {\tt plot.sl}


\paragraph{\index{plot.sl}plot.sl}
\begin{verbatim}
plot.sl - high level plotting functions
\end{verbatim}
Plot functions to allow quick visualization of Vamps output files
and matrix variables. The syntax of the high level functions ({\tt f\_plot}
and {\tt m\_plot}) is the same, no matter which underlying graphics method
is used. Low level functions depend on the graphics method used. The
following graphics methods can be used:
\begin{verbatim}
    plot        - gnu plotutils lib 1.1
    agl        - AGL graphics lib 3.?
    ext_plot    - using external gnu plotutils programs (UNIX only)    
    gnu_plot    - using gnuplot interface (UNIX only)    
              see also @#@ and @plt@.    
\end{verbatim}
Which of these methods can be used in this version of vamps is
printed to the screen when using the {\tt plot} command to initialize the
plot system. The {\tt show\_plotsys()} function can be used for this purpose
as well.


The {\tt outputdev} variable can be used to specify screen preview (1) or
hardcopy output (0).


If more than one graphics method is available you can switch between 
methods be setting the {\tt m\_plot} function to the method specific function.
Use the {\tt set\_plotsys} function to do so.

from file: {\tt plot.sl}


\paragraph{\index{plotstyle}plotstyle}
\begin{verbatim}
Int plotstyle;
\end{verbatim}
Set the plotting style:
\begin{verbatim}
0 = lines, 1 is markers, 2 = markers & lines

from file: @plot.sl@
\end{verbatim}


\paragraph{\index{set\_plotsys}set\_plotsys}
\begin{verbatim}
Void set_plotsys(Int nr)
\end{verbatim}
Sets the uses plotting system to the system {\tt nr}. It uses {\tt show\_plotsys}
to display the updated list of available plot-systems. If the {\tt plverb}
variable is $>$ 1

from file: {\tt plot.sl}


\paragraph{\index{set\_plotsys\_byname}set\_plotsys\_byname}
\begin{verbatim}
Void set_plotsys_byname(String name)
\end{verbatim}
Sets the uses plotting system to the system {\tt name}. It uses {\tt show\_plotsys}
to display the updated list of available plot-systems.

from file: {\tt plot.sl}


\paragraph{\index{show\_plotsys}show\_plotsys}
\begin{verbatim}
Void show_plotsys
\end{verbatim}
Prints short descriptions of all the installed plotting systems
to the screen. The system indicated with a '*' is the active system.

from file: {\tt plot.sl}


\paragraph{\index{plotpar}plotpar}
\begin{verbatim}
Void plotpar("parname")
\end{verbatim}
calls {\tt vsel} and {\tt graph} to plot parameter {\tt parname} from {\tt outfilename}.

from file: {\tt runut.sl}


\paragraph{\index{plotts}plotts}
\begin{verbatim}
Int plotts("tsname");
\end{verbatim}
plots a time serie from the start of the run to
current time 't' using the current plot settings


Returns: nothing

from file: {\tt runut.sl}


\paragraph{\index{savets}savets}
\begin{verbatim}
Int savets ("tsname", "fname", points)
\end{verbatim}
saves the time series tsname to the file fname from
point 0 to point points


Returns: -1 on failure, 0 on sucess

from file: {\tt runut.sl}


\paragraph{\index{v\_save\_all\_sets}v\_save\_all\_sets}
\begin{verbatim}
Void v_save_all_sets ()
\end{verbatim}
save all currently loaded datasets to a file. The
filenames are created by prepending 'set\_' to the name of
the dataset

from file: {\tt runut.sl}


\paragraph{\index{v\_show\_data\_sets}v\_show\_data\_sets}
\begin{verbatim}
Void v_show_data_sets ();
\end{verbatim}
shows all data set currently loaded


Returns: nothing

from file: {\tt runut.sl}


\paragraph{\index{adev}adev}
\begin{verbatim}
Array adev(Array mtx)
\end{verbatim}
Returns a row vector with the absolute mean deviations for
each column in {\tt mtx}.

from file: {\tt stats.sl}


\paragraph{\index{corr}corr}
\begin{verbatim}
Array corr(Array mtx)
\end{verbatim}
Calculate linear correlation coefficient (r) between x-data in
column 0 and y-data in subsequent columns.


Returns a row vector with correlation coefficients for each
column in {\tt mtx} w.r.t. column 0. Hence, Array[0,0] $=$ 1.

from file: {\tt stats.sl}


\paragraph{\index{covar}covar}
\begin{verbatim}
Array covar(Array mtx)
\end{verbatim}
Calculates the covariance between x-data in {\tt mtx} column 0 and
y-data in subsequent columns.


Returns a row vector with covariances for each column in {\tt mtx}
w.r.t. column 0. Hence, Array[0,0] contains the variance of the
data in the first column.

from file: {\tt stats.sl}


\paragraph{\index{linreg}linreg}
\begin{verbatim}
Array linreg(Array mtx)
\end{verbatim}
Calculates linear regression coefficients for straight line
fitting, i.e: y $=$ Ax + B. A is slope and B is y-intercept.
Argument is minimum 2*2 matrix, regression is performed for
all {\tt mtx} columns j$>$$=$1 (y-data) against column 0 (x-data).


Returns a 2*n matrix in which for each column j in {\tt mtx},
Array[0,j] $=$ A and Array[1,j] $=$ B. Hence, Array[0,0] $=$ 1 and
Array[1,0] $=$ 0.

from file: {\tt stats.sl}


\paragraph{\index{mean}mean}
\begin{verbatim}
Array mean(Array mtx)
\end{verbatim}
Returns a row vector with the mean values for each column
in {\tt mtx}.

from file: {\tt stats.sl}


\paragraph{\index{median}median}
\begin{verbatim}
Array median(Array mtx)
\end{verbatim}
Returns a row vector with the median values for each column
in {\tt mtx}.

from file: {\tt stats.sl}


\paragraph{\index{mmax}mmax}
\begin{verbatim}
Array mmax(Array mtx)
\end{verbatim}
Returns a row vector with the maximum values for each column
in {\tt mtx}.

from file: {\tt stats.sl}


\paragraph{\index{mmin}mmin}
\begin{verbatim}
Array mmin(Array mtx)
\end{verbatim}
Returns a row vector with the minimum values for each column
in {\tt mtx}.

from file: {\tt stats.sl}


\paragraph{\index{rmsq}rmsq}
\begin{verbatim}
Array rmsq(Array mtx)
\end{verbatim}
Returns a row vector with the root mean square or quadratic
mean values for each column in {\tt mtx}.

from file: {\tt stats.sl}


\paragraph{\index{sdev}sdev}
\begin{verbatim}
Array sdev(Array mtx)
\end{verbatim}
Returns a row vector with the standard deviations for each
column in {\tt mtx}.

from file: {\tt stats.sl}


\paragraph{\index{stats}stats}
\begin{verbatim}
Void stats
\end{verbatim}
Initialize statistical data analysis library. None of the
functions in {\tt stats.sl} can be used before this command is
issued.


All statistical data analysis is performed on matrices (2D
floating point arrays) in which the data are column-wise
organized.

from file: {\tt stats.sl}


\paragraph{\index{tstep\_top}tstep\_top}
\begin{verbatim}
tstep_top(tstepje)
\end{verbatim}
Calculates soilevaporation, potential transpiration, interception
and precipitation for each step at {\tt tstepje}.


Return order should be:
return(soilevap, pot-trans, intercep, precipitation);

from file: {\tt topsys.sl}


\paragraph{\index{f\_save}f\_save}
\begin{verbatim}
Void f_save(String var, String filename, String outfilename)
\end{verbatim}
Writes the variable {\tt var} from the vamps outputfile {\tt filename} to
an ascii file {\tt outfilename}. Depending on the nature of the variable
two ore more columns are created. The time data resides in the
first column.

from file: {\tt util.sl}


\paragraph{\index{usergetstr}usergetstr}
\begin{verbatim}
char *usergetstr(char *prompt);
\end{verbatim}
Gets a string from user, resets tty for use in interactive 
mode, the global variable userstr\_l gives the length of the 
user string. see also {\tt Slang\_reset\_tty} and {\tt SLang\_init\_tty}


If the first character of the string is a {\tt ':'} the remaining
part of the string is passed to {\tt system()};


If the first character of the string is a {\tt '?'} the remaining
part of the string is passed to {\tt vamps\_help()};
\begin{verbatim}
If the first character of the string is a '@' the remaining
\end{verbatim}
part of the string is passed to {\tt eval()};


Returns: string

from file: {\tt util.sl}


\paragraph{\index{vprof}vprof}
\begin{verbatim}
[Array], Int  vprof(String var, Float time, String filename)
\end{verbatim}
Returns -1 on failure or the number of rows
in the array and the array on success.


vprof reads a profile variable from the Vamps output file
{\tt filename} for a single timstep (time) in the simulation.
Depth is placed in column 1 and the value in column 0.
See also {\tt vsel}

from file: {\tt util.sl}


\paragraph{\index{vsel}vsel}
\begin{verbatim}
[array], Int vsel (parname, layer, filename)
\end{verbatim}
Returns -1 on failure or the number of rows
in the array and the array on success.
If {\tt layer $=$$=$ -1} then all layers are put into
the array.
Example:
\begin{verbatim}
variable ar;
if (vsel("volact",-1,"sat.out") != -1){
    ar = ();
}
\end{verbatim}
See also the stand-alone {\tt vsel} program and {\tt vprof}.

from file: {\tt util.sl}


\paragraph{\index{adim}adim}
\begin{verbatim}
Integer adim(Array arr)
Integer, ... asize(Array arr)
Integer atype(Array arr)
\end{verbatim}
Give information about array arr. adim returns the array
dimension (1, 2 or 3), asize returns the size for each
dimension and atype returns the data type of the array.

from file: {\tt vamps.sl}


\paragraph{\index{apropos}apropos}
\begin{verbatim}
apropos(String substring)
\end{verbatim}
apropos prints all functions and variables containing {\tt substring} to
stdout. It assumes the {\tt print()} intrinsic

from file: {\tt vamps.sl}


\paragraph{\index{asize}asize}
\begin{verbatim}
Integer adim(Array arr)
Integer, ... asize(Array arr)
Integer atype(Array arr)
\end{verbatim}
Give information about array arr. adim returns the array
dimension (1, 2 or 3), asize returns the size for each
dimension and atype returns the data type of the array.

from file: {\tt vamps.sl}


\paragraph{\index{at\_end}at\_end}
\begin{verbatim}
Void at_end()
\end{verbatim}
The vamps s-lang end function. Place extra stuff you want to
have done at the end of simulation in this function.

from file: {\tt vamps.sl}


\paragraph{\index{at\_start}at\_start}
\begin{verbatim}
Void at_start()
\end{verbatim}
This is the vamps s-lang startup function. It can be completely empty
if you want. The default shows all the data sets in memory.
Place extra stuff you want to
have done at startup in this function.

from file: {\tt vamps.sl}


\paragraph{\index{atype}atype}
\begin{verbatim}
Integer adim(Array arr)
Integer, ... asize(Array arr)
Integer atype(Array arr)
\end{verbatim}
Give information about array arr. adim returns the array
dimension (1, 2 or 3), asize returns the size for each
dimension and atype returns the data type of the array.

from file: {\tt vamps.sl}


\paragraph{\index{debug}debug}
\begin{verbatim}
Void debug()
\end{verbatim}
Alias for {\tt print\_stack}, prints the present contents of the S-Lang
stack to stdout

from file: {\tt vamps.sl}


\paragraph{\index{each\_step}each\_step}
\begin{verbatim}
Void each_step()
\end{verbatim}
The vamps S-Lang function that is executed after each timestep. 
Place extra stuff you want to have done after each timestep in 
this function. Normally this function is defined in {\tt vamps.sl} but
it can be redefined in any S-Lang file that is loaded by {\tt Vamps}.
It's default definition is as follows:
\begin{verbatim}
    define each_step ()
    {
       % Check for conditional switches to interactive mode, see stop.sl
       stop();
       v_printstr("CPU",string(cpu)); % Store CPU time used in output file
    }

from file: @vamps.sl@
\end{verbatim}


\paragraph{\index{echo}echo}
\begin{verbatim}
Void echo(obj)
\end{verbatim}
Prints an ascii representation of obj, followed by a newline
on standard output. Supported are: Character, Float, Integer,
String, and 1D and 2D Arrays for these types.

from file: {\tt vamps.sl}


\paragraph{\index{fprintf}fprintf}
\begin{verbatim}
Void printf(String fmt, ..., Integer N)
Void fprintf(Integer fp, String fmt, ..., Integer N)
String sprintf(String fmt, ..., Integer N)
\end{verbatim}
Formatted printing cf. the C printf family, except that
the number of parameters N must be given. printf and
fprintf return nothing, sprintf returns the formatted
string.

from file: {\tt vamps.sl}


\paragraph{\index{help}help}
\begin{verbatim}
Void help()
\end{verbatim}
Prints short info on the Vamps interactive mode. See also '?'

from file: {\tt vamps.sl}


\paragraph{\index{linspace}linspace}
\begin{verbatim}
Array linspace(Float x0, Float x1, Integer n)
\end{verbatim}
Returns a vector of {\tt n} elements with linearly spaced intervals
between {\tt x0} and {\tt x1}.

from file: {\tt vamps.sl}


\paragraph{\index{logspace}logspace}
\begin{verbatim}
Array logspace(Float x0, Float x1, Integer n)
\end{verbatim}
Returns a vector of {\tt n} elements with logarithmically spaced intervals
between 1E{\tt x0} and 1E{\tt x1}.

from file: {\tt vamps.sl}


\paragraph{\index{matrix}matrix}
\begin{verbatim}
Array matrix(Integer m, Integer n)
\end{verbatim}
Returns a matrix (two-dimensional floating point array)
with {\tt m} rows and {\tt n} columns. All elements are set to zero.
A row vector is created with {\tt m$=$1}, a column vector is created
with {\tt n$=$1}. Elements are addressed by [{\tt i,j}], where {\tt 0$<$$=$i$<$m}
and {\tt 0$<$$=$j$<$n}.

from file: {\tt vamps.sl}


\paragraph{\index{printf}printf}
\begin{verbatim}
Void printf(String fmt, ..., Integer N)
Void fprintf(Integer fp, String fmt, ..., Integer N)
String sprintf(String fmt, ..., Integer N)
\end{verbatim}
Formatted printing cf. the C printf family, except that
the number of parameters N must be given. printf and
fprintf return nothing, sprintf returns the formatted
string.

from file: {\tt vamps.sl}


\paragraph{\index{sprintf}sprintf}
\begin{verbatim}
Void printf(String fmt, ..., Integer N)
Void fprintf(Integer fp, String fmt, ..., Integer N)
String sprintf(String fmt, ..., Integer N)
\end{verbatim}
Formatted printing cf. the C printf family, except that
the number of parameters N must be given. printf and
fprintf return nothing, sprintf returns the formatted
string.

from file: {\tt vamps.sl}


\paragraph{\index{transpose}transpose}
\begin{verbatim}
Array transpose(Array mtx)
\end{verbatim}
Returns a transposed matrix of {\tt mtx}. The returned matrix is
a new one, the elements of {\tt mtx} are left unchanged.

from file: {\tt vamps.sl}


\paragraph{\index{ts\_echo}ts\_echo}
\begin{verbatim}
Void ts_echo(obj)
\end{verbatim}
Prints an ascii representation of obj, followed by a newline
on standard output. Supported are: Character, Float, Integer,
String, and 1D and 2D Arrays for these types. In this case
all floats and first columns of matrixes are regarded in {\tt ts(5)}
format

from file: {\tt vamps.sl}


\paragraph{\index{v\_run}v\_run}
\begin{verbatim}
Void v_run(String inf)
\end{verbatim}
Starts a Vamps run with input file {\tt inf}.

from file: {\tt vamps.sl}
