% Generated by sldoc.slx (1.14, (C) 1996, 1997) Tue Dec 16 10:54:34 1997


\paragraph{\index{\#}\#}
\begin{verbatim}
# [plotcommand]
\end{verbatim}
Operator to send a single command to the plotting system (gnuplot) when
followed by an argument or to enter plotting mode if no argument is given.
In plotting mode all commands should be gnuplot commands. To switch back to
normal mode give a single '\#'. The {\tt :} and \verb1\1@ operators also work within
plotting mode.


Some special characters can be used to addres S-Lang variables from within
plotting mode. Words withing \$'s are replaced by a string representation of
the S-Lang variable with that name. Example
\begin{verbatim}
variable h = "sin(x)";
# plot $h$
\end{verbatim}
To plot S-Lang matrix variables the \verb1\1@ char must be used. Example
\begin{verbatim}
variable m = mread("mydata.file",-1);
# plot @m@ using 1:2
\end{verbatim}
Although S-Lang matrix variables are zero based gnuplot starts counting at
1.


Variables from vamps output files can also be plotted. In this case both
the name of the of the variable and the output filename are needed.
They should be seperated by a colon and enclosed in \verb1\1@ characters:
\begin{verbatim}
# plot @precipitation:example1.out@ using boxes
\end{verbatim}
See also {\tt plt}.


from: {\tt ..$/$src$/$main$/$init\_s.c}

from file: {\tt intrin.doc}


\paragraph{\index{:}:}
\begin{verbatim}
: command [options ...]
\end{verbatim}
Execute a system {\tt command} in a subshell. Note that the
syntax does not use S-Lang conventions; everything after
the `{\tt :}' is passed to the shell. Returns nothing, a message
is printed on standard error if {\tt command} returned a non-zero
exit status.


from: {\tt ..$/$src$/$main$/$intrins.c}

from file: {\tt intrin.doc}


\paragraph{\index{?}?}
\begin{verbatim}
? [name ...]
\end{verbatim}
Invoke the {\tt vamps} online help system. This command can
only be used when running interactive.


If a single `{\tt ?}' is entered at the prompt, a list of
{\tt names}, typically {\tt vamps} command or variable names, for
which online documentation is present will be printed on
the screen.


If one or more {\tt name}(s) are given after the `{\tt ?}', the
available documentation text for those {\tt name}(s) is printed
on the screen.


If there is more than one screenfull of text, the {\tt cursor}
and {\tt PgDn$/$PgUp} keys will cause the text to scroll; pressing
`{\tt q}' will stop text display and go back to the command
prompt.


from: {\tt ..$/$src$/$main$/$intrins.c}

from file: {\tt intrin.doc}


\paragraph{\index{@}@}
\begin{verbatim}
@scriptname
\end{verbatim}
This operator passes it's argument to {\tt evalfile}. It
can only be used when running interactively. It
can be used to run scripts in stead of an expression
like:
\begin{verbatim}
() = evalfile("scriptname");
\end{verbatim}
Note the lack of a space between the operator and it's
argument!


from: {\tt ..$/$src$/$main$/$init\_s.c}

from file: {\tt intrin.doc}


\paragraph{\index{SLang\_init\_tty}SLang\_init\_tty}
\begin{verbatim}
Int SLang_init_tty()
\end{verbatim}
Initialize the S-Lang tty. 


from: {\tt ..$/$src$/$main$/$init\_s.c}

from file: {\tt intrin.doc}


\paragraph{\index{SLang\_reset\_tty}SLang\_reset\_tty}
\begin{verbatim}
Int SLang_reset_tty()
\end{verbatim}
Resets the S-Lang tty. Needed is you want to use stdin and
stdout from interactive mode.  


from: {\tt ..$/$src$/$main$/$init\_s.c}

from file: {\tt intrin.doc}


\paragraph{\index{Slai\_to\_s}Slai\_to\_s}
\begin{verbatim}
Float Slai_to_s(Float lai)
\end{verbatim}
Function called in the canopy module to convert lai to
canopy storage. If this S-Lang function is not defined the
lai value from the {\tt canopy} section in the input file is used.


from: {\tt ..$/$src$/$main$/$init\_s.c}

from file: {\tt intrin.doc}


\paragraph{\index{\_sets\_}\_sets\_}
\begin{verbatim}
Int _sets_
\end{verbatim}
Number of data sets in memory 


from: {\tt ..$/$src$/$main$/$init\_s.c}

from file: {\tt intrin.doc}


\paragraph{\index{\_version\_}\_version\_}
\begin{verbatim}
Int _version_
\end{verbatim}
Devide by 10 to get version number of Vamps 


from: {\tt ..$/$src$/$main$/$init\_s.c}

from file: {\tt intrin.doc}


\paragraph{\index{addset}addset}
\begin{verbatim}
Void addset(name, points)
\end{verbatim}
Adds an empty set {\tt name} with {\tt points} points
to the list


from: {\tt ..$/$src$/$main$/$init\_s.c}

from file: {\tt intrin.doc}


\paragraph{\index{addtohist}addtohist}
\begin{verbatim}
Int addtohist(string)
\end{verbatim}
Adds {\tt string} to the history list.


from: {\tt ..$/$src$/$main$/$init\_s.c}

from file: {\tt intrin.doc}


\paragraph{\index{arg0}arg0}
\begin{verbatim}
String arg0
\end{verbatim}
Readonly variable holding the name of the application
(``{\tt slash}'' if running interactive) or the name of the
current script file being processed.


from: {\tt ..$/$src$/$main$/$intrins.c}

from file: {\tt intrin.doc}


\paragraph{\index{closedef}closedef}
\begin{verbatim}
int closedef ()
\end{verbatim}
Description: closes a file previously opened with opendef
Returns: fclose's result


from: {\tt ..$/$src$/$deffile.lib$/$sl\_inter.c}

from file: {\tt intrin.doc}


\paragraph{\index{cont}cont}
\begin{verbatim}
cont()
\end{verbatim}
Continiue the current Vamps run 


from: {\tt ..$/$src$/$main$/$init\_s.c}

from file: {\tt intrin.doc}


\paragraph{\index{copyright}copyright}
\begin{verbatim}
Void disclaim
Void copyright
\end{verbatim}
Shows a short version of the GPL 


from: {\tt ..$/$src$/$main$/$init\_s.c}

from file: {\tt intrin.doc}


\paragraph{\index{cpu}cpu}
\begin{verbatim}
Float cpu()
\end{verbatim}
Return the number of CPU seconds used since the start of
the program. On msdos systems this will be equal to the
total run-time. On multitasking systems the actually used cpu time
will be returned.


from: {\tt ..$/$src$/$main$/$init\_s.c}

from file: {\tt intrin.doc}


\paragraph{\index{defverb}defverb}
\begin{verbatim}
Int defverb
\end{verbatim}
The verbose level in the input file routines. 0 is quit. A higher
the number makes this part of Vamps more verbose 


from: {\tt ..$/$src$/$deffile.lib$/$sl\_inter.c}

from file: {\tt intrin.doc}


\paragraph{\index{disclaim}disclaim}
\begin{verbatim}
Void disclaim
Void copyright
\end{verbatim}
Shows a short version of the GPL 


from: {\tt ..$/$src$/$main$/$init\_s.c}

from file: {\tt intrin.doc}


\paragraph{\index{error}error}
\begin{verbatim}
Void error(String str)
\end{verbatim}
Prints the text held in {\tt str} followed by a newline on standard
error output and raises internal error status.


from: {\tt ..$/$src$/$main$/$intrins.c}

from file: {\tt intrin.doc}


\paragraph{\index{exit}exit}
\begin{verbatim}
exit()
quit()
\end{verbatim}
Quits the S-Lang interpreter and returns to the
operating system 


from: {\tt ..$/$src$/$main$/$init\_s.c}

from file: {\tt intrin.doc}


\paragraph{\index{exit}exit}
\begin{verbatim}
exit()
quit()
\end{verbatim}
Quits the S-Lang interpreter and returns to the
operating system 


from: {\tt ..$/$src$/$main$/$init\_s.c}

from file: {\tt intrin.doc}


\paragraph{\index{format}format}
\begin{verbatim}
Void format(String fmt)
\end{verbatim}
Sets the floating point output format to {\tt fmt}. If given
as the nullstring (""), the default format is restored.
The default can be set by calling {\tt format} from {\tt .vampssl},
else it is set to ``{\tt \%6g}''. (Using the S-Lang intrinsic
{\tt set\_float\_format} is deprecated.)


from: {\tt ..$/$src$/$main$/$intrins.c}

from file: {\tt intrin.doc}


\paragraph{\index{fscanf}fscanf}
\begin{verbatim}
..., Integer scanf(String *fmt)
..., Integer fscanf(Integer fid, String *fmt)
..., integer sscanf(String *str, String *fmt)
\end{verbatim}
Formatted variable assignment like their C counterparts
(see {\tt scanf(3)}), except that the assigned values, followed
by the number of assignments are placed on the stack. For
example:
\begin{verbatim}
    variable month, day;
    if(sscanf("December, 5\n", "%[A-Z a-z], %d\n") == 2)
    {
        (month, day) = ();
        ...
    }
\end{verbatim}
pushes a string followed by an integer on the stack.


from: {\tt ..$/$src$/$main$/$intrins.c}

from file: {\tt intrin.doc}


\paragraph{\index{getarg}getarg}
\begin{verbatim}
String getarg()
\end{verbatim}
Returns a string holding the next command line argument or
``{\tt --}'' if there are no options left. Increases the internal
index to the command line arguments each time it is called.
For example:
\begin{verbatim}
    variable arg;
    while(arg = getarg(), strcmp("--", arg))
    {
        ...
    }
\end{verbatim}
from: {\tt ..$/$src$/$main$/$intrins.c}

from file: {\tt intrin.doc}


\paragraph{\index{getdefar}getdefar}
\begin{verbatim}
String getdefstr(section, name, def, fname, exitonerror)
Int getdefint(section, name, def, fname, exitonerror)
Float getdefdoub(section, name, def, fname, exitonerror)
Int, Array getdefar(section, name, fname, exitonerror)
\end{verbatim}
A series of functions te get information from Vamps input$/$output files.


They get the value for variable {\tt name} in section {\tt section}
of the file {\tt fname}. If {\tt exitonerror} !$=$ 0 then the
program is terminated if {\tt name} is not found in {\tt section}.
If {\tt exitonerror} $=$$=$ 0 and {\tt name} is not found {\tt def} is
returned instead.
{\tt getdefar} is somewhat different. Is does not allow a {\tt def}
variable and return either one or two values. It always returns
the number of points read. If zero
points are read no array is returned.


Say we use the following file (named ex.inp):
\begin{verbatim}
[example]
examplename = nonsense
\end{verbatim}
And then call the {\tt getdefstr} function like this:
\begin{verbatim}
variable exn = getdefstr("example","examplename","Not found","ex.inp",0);
\end{verbatim}
the variable exn will now hold the string "nonsense".


from: {\tt ..$/$src$/$deffile.lib$/$sl\_inter.c}

from file: {\tt intrin.doc}


\paragraph{\index{getdefdoub}getdefdoub}
\begin{verbatim}
String getdefstr(section, name, def, fname, exitonerror)
Int getdefint(section, name, def, fname, exitonerror)
Float getdefdoub(section, name, def, fname, exitonerror)
Int, Array getdefar(section, name, fname, exitonerror)
\end{verbatim}
A series of functions te get information from Vamps input$/$output files.


They get the value for variable {\tt name} in section {\tt section}
of the file {\tt fname}. If {\tt exitonerror} !$=$ 0 then the
program is terminated if {\tt name} is not found in {\tt section}.
If {\tt exitonerror} $=$$=$ 0 and {\tt name} is not found {\tt def} is
returned instead.
{\tt getdefar} is somewhat different. Is does not allow a {\tt def}
variable and return either one or two values. It always returns
the number of points read. If zero
points are read no array is returned.


Say we use the following file (named ex.inp):
\begin{verbatim}
[example]
examplename = nonsense
\end{verbatim}
And then call the {\tt getdefstr} function like this:
\begin{verbatim}
variable exn = getdefstr("example","examplename","Not found","ex.inp",0);
\end{verbatim}
the variable exn will now hold the string "nonsense".


from: {\tt ..$/$src$/$deffile.lib$/$sl\_inter.c}

from file: {\tt intrin.doc}


\paragraph{\index{getdefint}getdefint}
\begin{verbatim}
String getdefstr(section, name, def, fname, exitonerror)
Int getdefint(section, name, def, fname, exitonerror)
Float getdefdoub(section, name, def, fname, exitonerror)
Int, Array getdefar(section, name, fname, exitonerror)
\end{verbatim}
A series of functions te get information from Vamps input$/$output files.


They get the value for variable {\tt name} in section {\tt section}
of the file {\tt fname}. If {\tt exitonerror} !$=$ 0 then the
program is terminated if {\tt name} is not found in {\tt section}.
If {\tt exitonerror} $=$$=$ 0 and {\tt name} is not found {\tt def} is
returned instead.
{\tt getdefar} is somewhat different. Is does not allow a {\tt def}
variable and return either one or two values. It always returns
the number of points read. If zero
points are read no array is returned.


Say we use the following file (named ex.inp):
\begin{verbatim}
[example]
examplename = nonsense
\end{verbatim}
And then call the {\tt getdefstr} function like this:
\begin{verbatim}
variable exn = getdefstr("example","examplename","Not found","ex.inp",0);
\end{verbatim}
the variable exn will now hold the string "nonsense".


from: {\tt ..$/$src$/$deffile.lib$/$sl\_inter.c}

from file: {\tt intrin.doc}


\paragraph{\index{getdefstr}getdefstr}
\begin{verbatim}
String getdefstr(section, name, def, fname, exitonerror)
Int getdefint(section, name, def, fname, exitonerror)
Float getdefdoub(section, name, def, fname, exitonerror)
Int, Array getdefar(section, name, fname, exitonerror)
\end{verbatim}
A series of functions te get information from Vamps input$/$output files.


They get the value for variable {\tt name} in section {\tt section}
of the file {\tt fname}. If {\tt exitonerror} !$=$ 0 then the
program is terminated if {\tt name} is not found in {\tt section}.
If {\tt exitonerror} $=$$=$ 0 and {\tt name} is not found {\tt def} is
returned instead.
{\tt getdefar} is somewhat different. Is does not allow a {\tt def}
variable and return either one or two values. It always returns
the number of points read. If zero
points are read no array is returned.


Say we use the following file (named ex.inp):
\begin{verbatim}
[example]
examplename = nonsense
\end{verbatim}
And then call the {\tt getdefstr} function like this:
\begin{verbatim}
variable exn = getdefstr("example","examplename","Not found","ex.inp",0);
\end{verbatim}
the variable exn will now hold the string "nonsense".


from: {\tt ..$/$src$/$deffile.lib$/$sl\_inter.c}

from file: {\tt intrin.doc}


\paragraph{\index{hist}hist}
\begin{verbatim}
Void hist()
\end{verbatim}
Prints the history of command entered at the interactive
prompt to the screen.


Note that the  help operator `?' is never added to the list.
The excecute operator \verb1\1{\tt  is replaced by }() $=$ evalfile("arg");@.


from: {\tt ..$/$src$/$main$/$init\_s.c}

from file: {\tt intrin.doc}


\paragraph{\index{interpreter}interpreter}
\begin{verbatim}
Void interpreter(int verb)
\end{verbatim}
Start the Vamps interactive mode. If {\tt verb} !$=$ 0
the opening banner will be shown.


from: {\tt ..$/$src$/$main$/$init\_s.c}

from file: {\tt intrin.doc}


\paragraph{\index{makeindex}makeindex}
\begin{verbatim}
Int makeindex(filename)
\end{verbatim}
Makes an index (in memory) of file {\tt filename}. It is usefull
to speed up random access to large vamps output files. Use
{\tt saveindex} to save the index to a file.


from: {\tt ..$/$src$/$deffile.lib$/$sl\_inter.c}

from file: {\tt intrin.doc}


\paragraph{\index{maqend}maqend}
\begin{verbatim}
Void maqend(int maq_m)
\end{verbatim}
Cleans up after {\tt maqinit}


from: {\tt ..$/$src$/$maq.lib$/$marquard.c}

from file: {\tt intrin.doc}


\paragraph{\index{maqhead}maqhead}
\begin{verbatim}
Void maqhead()
\end{verbatim}
prints startup options of the {\tt maqrun} procedure to
stderr


from: {\tt ..$/$src$/$maq.lib$/$marquard.c}

from file: {\tt intrin.doc}


\paragraph{\index{maqinit}maqinit}
\begin{verbatim}
Int maqinit(int m, int n)
\end{verbatim}
Initializes the fitting system. n is the number of
parameters to use, m the number of points used in the
fitting process.


Returns 0 on success, -1 on error


from: {\tt ..$/$src$/$maq.lib$/$marquard.c}

from file: {\tt intrin.doc}


\paragraph{\index{maqrun}maqrun}
\begin{verbatim}
Int maqrun()
\end{verbatim}
Runs the fitting procedure with the current settings
The S-Lang function {\tt Int fitf(Array par[], Int m, Int n)} which
sets the {\tt rv[]} array should be defined and working. It should return
0 if successfull and 1 if the function could not be resolved with
the supplied values for {\tt par[]}.


Returns 0 on success


from: {\tt ..$/$src$/$maq.lib$/$marquard.c}

from file: {\tt intrin.doc}


\paragraph{\index{maqtail}maqtail}
\begin{verbatim}
Void maqtail()
\end{verbatim}
prints exit info of the {\tt maqrun} procedure to
stderr


from: {\tt ..$/$src$/$maq.lib$/$marquard.c}

from file: {\tt intrin.doc}


\paragraph{\index{mread}mread}
\begin{verbatim}
Array mread(String file, Integer nc)
\end{verbatim}
Reads a matrix (2D floating point array) from the named
ascii {\tt file}, which is supposed to consist of lines with
numerical fields separated by whitespace (spaces, TABS
and$/$or newlines). Lines starting with `{\tt \#}' or `{\tt \%}' are
considered comments; blank lines are skipped. If {\tt nc}$>$0,
the number of matrix columns are set to {\tt nc}. If {\tt nc} is
zero or less, the number of columns are set equal to the
number of fields in the input lines. The number of matrix
rows are calculated from dividing the number of values by
the number of lines in the input {\tt file}.


from: {\tt ..$/$src$/$main$/$intrins.c}

from file: {\tt intrin.doc}


\paragraph{\index{mwrite}mwrite}
\begin{verbatim}
Void mwrite(Array m, String fname)
\end{verbatim}
Writes the double matrix {\tt m} to an ascii file
Returns nothing;


from: {\tt ..$/$src$/$main$/$intrins.c}

from file: {\tt intrin.doc}


\paragraph{\index{opendef}opendef}
\begin{verbatim}
Int opendef (String fname)
\end{verbatim}
open a file for processing, close with closedef()


Returns: 0 on error, otherwise 1


Remarks: Opendef is used to speed up processing of files that are used
in a \_sequential\_ way. If files must be accessed randomely {\tt rinimem}
or {\tt readindex} should be used


from: {\tt ..$/$src$/$deffile.lib$/$sl\_inter.c}

from file: {\tt intrin.doc}


\paragraph{\index{pause}pause}
\begin{verbatim}
Void pause(Integer n)
\end{verbatim}
Suspends processing for {\tt n} seconds, if {\tt n} is positive. If
{\tt n} is less than or equals zero, processing waits until a key
is pressed on the keyboard for which the user is prompted on
standard error.


from: {\tt ..$/$src$/$main$/$intrins.c}

from file: {\tt intrin.doc}


\paragraph{\index{plt}plt}
\begin{verbatim}
Void plt_close()
\end{verbatim}
Closes the pipe to the plotting program


from: {\tt ..$/$src$/$main$/$init\_s.c}

from file: {\tt intrin.doc}


\paragraph{\index{plt\_close}plt\_close}
\begin{verbatim}
Void plt_close()
\end{verbatim}
Closes the pipe to the plotting program


from: {\tt ..$/$src$/$main$/$init\_s.c}

from file: {\tt intrin.doc}


\paragraph{\index{prompt}prompt}
\begin{verbatim}
String prompt(String str1, String str2)
\end{verbatim}
Sets the prompts used in interactive mode. Default prompts
are ``{\tt Slash$>$ }'' and ``{\tt \_Slash$>$ }''. If either argument
is the nullstring ({\tt ""}), that prompt is not changed. The
sequences {\tt \%\#} and {\tt \%\$} are translated to the current input
line number and stack depth respectively.


from: {\tt ..$/$src$/$main$/$intrins.c}

from file: {\tt intrin.doc}


\paragraph{\index{quit}quit}
\begin{verbatim}
exit()
quit()
\end{verbatim}
Quits the S-Lang interpreter and returns to the
operating system 


from: {\tt ..$/$src$/$main$/$init\_s.c}

from file: {\tt intrin.doc}


\paragraph{\index{readindex}readindex}
\begin{verbatim}
Int readindex(filename)
\end{verbatim}
Reads an index previously save with {\tt saveindex}.


from: {\tt ..$/$src$/$deffile.lib$/$sl\_inter.c}

from file: {\tt intrin.doc}


\paragraph{\index{readset}readset}
\begin{verbatim}
Int readset(String filename, String setname)
\end{verbatim}
Reads set {\tt setname} from file {\tt fname}. Returns the
id of the set or -1 on error.


from: {\tt ..$/$src$/$main$/$init\_s.c}

from file: {\tt intrin.doc}


\paragraph{\index{save\_history}save\_history}
\begin{verbatim}
Int save_history(fname)
\end{verbatim}
Save the command entered at the interactive prompt to the
file {\tt fname}. Returns -1 on error (this file could not be
opened in write mode) and 0 on success.


from: {\tt ..$/$src$/$main$/$init\_s.c}

from file: {\tt intrin.doc}


\paragraph{\index{saveindex}saveindex}
\begin{verbatim}
Int saveindex(filename)
\end{verbatim}
Save the index to the file {\tt filename}.


from: {\tt ..$/$src$/$deffile.lib$/$sl\_inter.c}

from file: {\tt intrin.doc}


\paragraph{\index{scanf}scanf}
\begin{verbatim}
..., Integer scanf(String *fmt)
..., Integer fscanf(Integer fid, String *fmt)
..., integer sscanf(String *str, String *fmt)
\end{verbatim}
Formatted variable assignment like their C counterparts
(see {\tt scanf(3)}), except that the assigned values, followed
by the number of assignments are placed on the stack. For
example:
\begin{verbatim}
    variable month, day;
    if(sscanf("December, 5\n", "%[A-Z a-z], %d\n") == 2)
    {
        (month, day) = ();
        ...
    }
\end{verbatim}
pushes a string followed by an integer on the stack.


from: {\tt ..$/$src$/$main$/$intrins.c}

from file: {\tt intrin.doc}


\paragraph{\index{scrcols}scrcols}
\begin{verbatim}
Integer scrcols
Integer scrrows
\end{verbatim}
Readonly variables set to the number of (text) columns and
rows of the screen. These values change when the window
size changes.


from: {\tt ..$/$src$/$main$/$intrins.c}

from file: {\tt intrin.doc}


\paragraph{\index{scrrows}scrrows}
\begin{verbatim}
Integer scrcols
Integer scrrows
\end{verbatim}
Readonly variables set to the number of (text) columns and
rows of the screen. These values change when the window
size changes.


from: {\tt ..$/$src$/$main$/$intrins.c}

from file: {\tt intrin.doc}


\paragraph{\index{setidx}setidx}
\begin{verbatim}
void sl_setidx(char *idxname);
\end{verbatim}
from: {\tt ..$/$src$/$main$/$intrins.c}

from file: {\tt intrin.doc}


\paragraph{\index{sl\_met\_init}sl\_met\_init}
\begin{verbatim}
Void sl_met_init
\end{verbatim}
Initializes the meteo functions:
{\tt eaes, vslope, e0b}


from: {\tt ..$/$src$/$main$/$init\_s.c}

from file: {\tt intrin.doc}


\paragraph{\index{slpath}slpath}
\begin{verbatim}
String slpath(String str)
\end{verbatim}
Sets a colon (`{\tt :}') separated pathlist of directories to
search when loading S-Lang command files (using {\tt autoload}
and {\tt evalfile}).  If {\tt str} is the nullstring ({\tt ""}), the
pathlist is not changed. The returned string holds the
previous pathlist.


Within a pathlist specification, a single dot (`{\tt .}') denotes
the current working directory. If the first character of
a path in the list is the tilde (`{\tt \verb1~1}'), it is expanded to
the home directory obtained from the {\tt HOME} environment
variable. For example, if the user maintains a library of
S-Lang command files in a directory {\tt \$HOME$/$lib$/$vamps},
the command:
\begin{verbatim}
    () = slpath(strcat("~/lib/vamps:", slpath("")));
\end{verbatim}
can be placed in the {\tt .vampssl} file in the user's home
directory. Script files, however, should restore the
previous pathlist before terminating.


from: {\tt ..$/$src$/$main$/$intrins.c}

from file: {\tt intrin.doc}


\paragraph{\index{sreadline}sreadline}
\begin{verbatim}
Int sreadline
\end{verbatim}
If this variable is 1 vamps runs in interactive mode (readline
initialized.). If this is set to 0 pressing \verb1^1C causes Vamps to
switch to interactive mode on unix systems. 


from: {\tt ..$/$src$/$main$/$init\_s.c}

from file: {\tt intrin.doc}


\paragraph{\index{sscanf}sscanf}
\begin{verbatim}
..., Integer scanf(String *fmt)
..., Integer fscanf(Integer fid, String *fmt)
..., integer sscanf(String *str, String *fmt)
\end{verbatim}
Formatted variable assignment like their C counterparts
(see {\tt scanf(3)}), except that the assigned values, followed
by the number of assignments are placed on the stack. For
example:
\begin{verbatim}
    variable month, day;
    if(sscanf("December, 5\n", "%[A-Z a-z], %d\n") == 2)
    {
        (month, day) = ();
        ...
    }
\end{verbatim}
pushes a string followed by an integer on the stack.


from: {\tt ..$/$src$/$main$/$intrins.c}

from file: {\tt intrin.doc}


\paragraph{\index{strv}strv}
\begin{verbatim}
Array strv(String str, String sep)
\end{verbatim}
Splits the string {\tt str} into fields separated by one or
more characters from the string {\tt sep}. A terminating
newline is always considered a field separator. Returns a
vector (1D array) of strings set to the individual fields.
For example:
\begin{verbatim}
    variable fid, i, s, v;
    fid = fopen("my_file.dat", "r");
    while(fgets(fid) > 0)
    {
        s = ();
        v = strv(s, " \t");
        for(i = 0; i < asize(v); i++)
            printf("%16s\n", v[i], 1);
    }
\end{verbatim}
reads an entire data file of fields separated by runs of
whitespace characters and prints the fields on standard
output.


from: {\tt ..$/$src$/$main$/$intrins.c}

from file: {\tt intrin.doc}


\paragraph{\index{tty}tty}
\begin{verbatim}
Integer tty
\end{verbatim}
Readonly variable set to {\tt 1} if standard output is a tty, {\tt 2}
if standard input is a tty, {\tt 3} if both standard input and
output are connected to a tty, else set to {\tt 0}.


from: {\tt ..$/$src$/$main$/$intrins.c}

from file: {\tt intrin.doc}


\paragraph{\index{v\_cumbot}v\_cumbot}
\begin{verbatim}
Float v_cumbot
\end{verbatim}
Cumulative flow trough bottom of profile calculated from start of
simulation (Read only)


from: {\tt ..$/$src$/$main$/$init\_s.c}

from file: {\tt intrin.doc}


\paragraph{\index{v\_cumeva}v\_cumeva}
\begin{verbatim}
Float v_cumeva
\end{verbatim}
Cumulative evaporation (from wet canopy) (Read only) 


from: {\tt ..$/$src$/$main$/$init\_s.c}

from file: {\tt intrin.doc}


\paragraph{\index{v\_cumintc}v\_cumintc}
\begin{verbatim}
Float v_cumint
\end{verbatim}
Cumulative interception [cm] (Read only) 


from: {\tt ..$/$src$/$main$/$init\_s.c}

from file: {\tt intrin.doc}


\paragraph{\index{v\_cumprec}v\_cumprec}
\begin{verbatim}
Float v_cumprec
\end{verbatim}
Cumulative precipitation [cm] (Read only) 


from: {\tt ..$/$src$/$main$/$init\_s.c}

from file: {\tt intrin.doc}


\paragraph{\index{v\_cumtop}v\_cumtop}
\begin{verbatim}
Float v_cumtop
\end{verbatim}
Cumulative flow trough top of profile calculated from start of
simulation (Read only)


from: {\tt ..$/$src$/$main$/$init\_s.c}

from file: {\tt intrin.doc}


\paragraph{\index{v\_cumtra}v\_cumtra}
\begin{verbatim}
Float v_cumtra
\end{verbatim}
Cumulative transpiration. This is equal to {\tt v\_rootextract} if
transpiration reduction is not calculated (Read only) 


from: {\tt ..$/$src$/$main$/$init\_s.c}

from file: {\tt intrin.doc}


\paragraph{\index{v\_dt}v\_dt}
\begin{verbatim}
Float v_dt
\end{verbatim}
Current timestep in interation process. The 'real' timeste can
be found in the {\tt v\_thiststep} variable (in days) 


from: {\tt ..$/$src$/$main$/$init\_s.c}

from file: {\tt intrin.doc}


\paragraph{\index{v\_getspar}v\_getspar}
\begin{verbatim}
Float v_getspar(int nr, char *des)
\end{verbatim}
Gets parameter from soil description structure


from: {\tt ..$/$src$/$main$/$init\_s.c}

from file: {\tt intrin.doc}


\paragraph{\index{v\_masbal}v\_masbal}
\begin{verbatim}
Float v_masbal
\end{verbatim}
mass balance error in \%. Read only.


from: {\tt ..$/$src$/$main$/$init\_s.c}

from file: {\tt intrin.doc}


\paragraph{\index{v\_postsoil}v\_postsoil}
\begin{verbatim}
Void v_postsoil
\end{verbatim}
Cleans up after the soil module


from: {\tt ..$/$src$/$main$/$init\_s.c}

from file: {\tt intrin.doc}


\paragraph{\index{v\_presoil}v\_presoil}
\begin{verbatim}
Void v_presoil()
\end{verbatim}
Initialize the soil moisture module


from: {\tt ..$/$src$/$main$/$init\_s.c}

from file: {\tt intrin.doc}


\paragraph{\index{v\_printstr}v\_printstr}
\begin{verbatim}
Void v_printstr(name, str)
\end{verbatim}
Stores variable {\tt str} with name {\tt name} in the output-file.
This function can be used in the {\tt each\_step} function only.
It's purpose is to store extra information in the output-file.


from: {\tt ..$/$src$/$main$/$init\_s.c}

from file: {\tt intrin.doc}


\paragraph{\index{v\_printsum}v\_printsum}
\begin{verbatim}
Void v_printsum()
\end{verbatim}
Prints actual water balance summary at {\tt stderr} 


from: {\tt ..$/$src$/$main$/$init\_s.c}

from file: {\tt intrin.doc}


\paragraph{\index{v\_qbot}v\_qbot}
\begin{verbatim}
Float v_qbot
\end{verbatim}
Flow trough bottom of profile for current timestep (Read only)


from: {\tt ..$/$src$/$main$/$init\_s.c}

from file: {\tt intrin.doc}


\paragraph{\index{v\_qtop}v\_qtop}
\begin{verbatim}
Float v_qtop
\end{verbatim}
Flow trough top of profile for current timestep (Read only) 


from: {\tt ..$/$src$/$main$/$init\_s.c}

from file: {\tt intrin.doc}


\paragraph{\index{v\_rootextract}v\_rootextract}
\begin{verbatim}
Float v_rootextract
\end{verbatim}
Cumulative amount of water extracted by the roots (actual
transpiration) (Read only) 


from: {\tt ..$/$src$/$main$/$init\_s.c}

from file: {\tt intrin.doc}


\paragraph{\index{v\_setspar}v\_setspar}
\begin{verbatim}
Void v_setspar(int nr, char *des, val double)
\end{verbatim}
set parameter in soil description structure


from: {\tt ..$/$src$/$main$/$init\_s.c}

from file: {\tt intrin.doc}


\paragraph{\index{v\_smd}v\_smd}
\begin{verbatim}
Float v_smd
\end{verbatim}
Soil moisture deficit for current timestep in cm 


from: {\tt ..$/$src$/$main$/$init\_s.c}

from file: {\tt intrin.doc}


\paragraph{\index{v\_steps}v\_steps}
\begin{verbatim}
Int v_steps
\end{verbatim}
Number of steps in the current simulation (Read,only) 


from: {\tt ..$/$src$/$main$/$init\_s.c}

from file: {\tt intrin.doc}


\paragraph{\index{v\_t}v\_t}
\begin{verbatim}
Float v_t
\end{verbatim}
Current time in simulation (in days) 


from: {\tt ..$/$src$/$main$/$init\_s.c}

from file: {\tt intrin.doc}


\paragraph{\index{v\_volact}v\_volact}
\begin{verbatim}
Float v_volact
\end{verbatim}
Actual water content of the total soil profile [cm] (Read only)


from: {\tt ..$/$src$/$main$/$init\_s.c}

from file: {\tt intrin.doc}


\paragraph{\index{v\_volsat}v\_volsat}
\begin{verbatim}
Float v_volsat
\end{verbatim}
Water content of the soil profile at saturation (read only)


from: {\tt ..$/$src$/$main$/$init\_s.c}

from file: {\tt intrin.doc}


\paragraph{\index{v\_vpd}v\_vpd}
\begin{verbatim}
Float v_vpd
\end{verbatim}
Actual vapour pressure deficit in mbar (Read only)


from: {\tt ..$/$src$/$main$/$init\_s.c}

from file: {\tt intrin.doc}


\paragraph{\index{vamps\_help}vamps\_help}
Void vamps\_help(String key)


gets help on {\tt key}. See {\tt '?'} for further info.


from: {\tt ..$/$src$/$main$/$init\_s.c}

from file: {\tt intrin.doc}


\paragraph{\index{verbose}verbose}
\begin{verbatim}
Int verbose
\end{verbatim}
The general verbose level in the current simulation. 0 is quiet. A
higher the number makes Vamps more verbose. Verbose levels above 1
are usefull for debugging only 


from: {\tt ..$/$src$/$main$/$init\_s.c}

from file: {\tt intrin.doc}


\paragraph{\index{version}version}
\begin{verbatim}
Integer version
\end{verbatim}
Readonly variable holding the current {\tt slash} version number
multiplied by 100.


from: {\tt ..$/$src$/$main$/$intrins.c}

from file: {\tt intrin.doc}


\paragraph{\index{writememini}writememini}
\begin{verbatim}
Int writememini(fname, inifname)
\end{verbatim}
Writes the file read into memory with the {\tt rinmem} function
to a file. If {\tt fname} is "-" output is send to stdout.
inifname if the name of the file in memory.


from: {\tt ..$/$src$/$deffile.lib$/$sl\_inter.c}

from file: {\tt intrin.doc}
